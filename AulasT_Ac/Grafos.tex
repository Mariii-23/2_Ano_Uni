% Created 2021-02-04 Thu 12:11
% Intended LaTeX compiler: pdflatex
\documentclass[11pt]{article}
\usepackage[utf8]{inputenc}
\usepackage[T1]{fontenc}
\usepackage{graphicx}
\usepackage{grffile}
\usepackage{longtable}
\usepackage{wrapfig}
\usepackage{rotating}
\usepackage[normalem]{ulem}
\usepackage{amsmath}
\usepackage{textcomp}
\usepackage{amssymb}
\usepackage{capt-of}
\usepackage{hyperref}
\date{\today}
\title{Grafos}
\hypersetup{
 pdfauthor={},
 pdftitle={Grafos},
 pdfkeywords={},
 pdfsubject={},
 pdfcreator={Emacs 27.1 (Org mode 9.5)}, 
 pdflang={English}}
\begin{document}

\maketitle
\tableofcontents

\section{Tipos de grafos}
\label{sec:org40f4aae}
\subsection{Grafos Orientados}
\label{sec:orgcb89a62}
Um grafo orientado é um par (V,E) :
\begin{itemize}
\item V : Um conjunto finito de vértices ou nós
\item E : Relaçáo binária sobre V (o conjunto de arestas ou arcos do grafo)
\end{itemize}

\subsubsection{Exemplo de um grafo}
\label{sec:org86d08e5}
\href{./exemplo\_grafo.png}{Exemplo de um grafo
}
\begin{itemize}
\item Se (i,j)\(\in\) E, j diz-se adjacente a i
i e j são respectivamente os vértices de origem e destino da aresta (i,j)

\item O grau de entrada de um vértice  é o número de arestas com destino nesse vértice
\item O grau de saída de um vértice  é o número de arestas com origem nesse vértice
\item Uma aresta (i,i) designa-se por anel
\item O número máximo de arestas de um grafo com conjunto de vértices V é V\textsuperscript{2}
\item Um grafo diz-se esparso se o número de arestas for muito inferior a V\textsuperscript{2}, e denso em caso contrário
\end{itemize}

\subsubsection{Representação em Computador de Grafos Orientados}
\label{sec:org4d78884}
\href{./matrix\_grafo.png}{Matrix Grafo}

\begin{verbatim}
#define MAX 100
typedef char GraphM[MAX][MAX];
\end{verbatim}

\begin{enumerate}
\item Tempo
\label{sec:org77f358a}
\begin{itemize}
\item O espaço de memória necessário para representar um grafo (V,E) é \(\Theta\)(V\textsuperscript{2}),  independente do número de arcos, e portanto da densidade do grafo;
\item é poss+ivel verificar em tempo constante se dois vértices são adjacentes
\item para conhecer os  adjacentes a um determinado vértice u, é necessário percorrer todos os vértices v do grafo, consultando para cada um a posição (u,v) da matriz.
\end{itemize}
\end{enumerate}

\subsubsection{Representação por Listas de Adjacências}
\label{sec:org7bd8bf0}
Uma representação alternativa consiste em associar a cada vértice do grafo uma lista contendo os vértices que lhe são adjacentes.

\begin{itemize}
\item um array de apontadores
\item cujos índices correspondem aos vértices do grafo, e
\item contendo em cada posição o (endereço do) primeiro elemento da lista de adjacência do vértice respectivo
\end{itemize}

\href{./lista\_grafo.png}{Exemplo}

\begin{verbatim}
#define MAX 100

struct edge {
  int dest;
  struct edge *next;
};
typedef struct edge *GraphL[MAX];
\end{verbatim}

\begin{enumerate}
\item Tempo
\label{sec:org4b3c5c8}
\begin{itemize}
\item O espaço de memória necessário para esta representação de um grafo (V,E)  é  \(\Theta\)(V+E), o que a torna uma representação eficiente no caso de grafos esparsos;
\item o teste de adjacência de dois vértices obriga à travessia de uma lista ligada, executada no pior caso em tempo \(\Theta\)(V);
\item a consulta dos adjacentes a um vértice u implica apenas percorrer a lista de adjacências de u, em vez de percorrer todos os vértices. Num grafo pouco denso, o impacto na implementação de um algoritmo que efectue esta operação repetidamente pode ser muito grande. É o caso por exemplo dos algoritmos de travessia de grafos.
\end{itemize}
\end{enumerate}

\subsection{Grafos com pesos}
\label{sec:org633bfdc}
Em certos contextos é útil associar informação (pesos) às arestas de um grafo, em particular numérica.
\subsubsection{Representação em Computador de Grafos com Pesos}
\label{sec:org416aa74}
\begin{enumerate}
\item Matriz
\label{sec:org12c4f32}
\href{./peso\_grafo.png}{Exemplo}

Em C:
\begin{verbatim}
#define NE 0 // Nodo vazio
#define MAX 100
typedef int WEIGHT;
typedef WEIGHT GraphM[MAX][MAX];
\end{verbatim}
\end{enumerate}
\subsubsection{Representação por Listas de Adjacências}
\label{sec:orga915417}
\begin{itemize}
\item As arestas do grafo são aqui representadas por nós das listas ligadas de adjacências. Sendo assim, basta criar um campo adicional nestas estruturas (nós das listas) para guardar o peso das arestas.

\item Note-se que, uma vez que nesta representação o teste de existência de uma aresta não é feito pela consulta do valor numérico do peso (como era o caso com uma matriz de adjacências), mas sim pela existência ou não de um nó com um determinado destino na lista ligada, não é agora necessária a utilização de um valor especial NE para representar o peso de uma aresta inexistente
\end{itemize}

Em C:
\begin{verbatim}
#define NE 0 // Valor de uma aresta inexistente
#define MAX 100
typedef int WEIGHT;
struct edge {
  int dest;
  WEIGHT weight;
  struct edge *next;
};
typedef struct edge *GraphL[MAX];
\end{verbatim}

\subsection{Grafos não orientados}
\label{sec:orgb13d44a}
Num grafo não-orientado as arestas são conjuntos com dois vértices $\backslash${u,v$\backslash$} \(\in\) E em vez de pares ordenados. Por outras palavras, as aresta são bi-direccionais, o que é adequado, por exemplo, para modelar redes em que todas as ligações entre pares de vértices funcionam nos dois sentidos.
\href{./nao\_orientado\_grafo.png}{Exemplo}

\subsubsection{Representação em Computador de Grafos Não-orientados}
\label{sec:orgb61ab0f}
A representação típica de um grafo não-orientado passa pela sua conversão para um grafo orientado simétrico, em que se $\backslash${u,v$\backslash$} \(\in\) E então também $\backslash${v,u$\backslash$} \(\in\) E.

Note-se que uma tal representação contém redundância:


\begin{itemize}
\item No caso da representação por uma matriz de adjacências poder-se-á eliminar esta redundância representando de forma eficiente apenas uma matriz triangular.
\end{itemize}


\begin{itemize}
\item No caso da representação por listas de adjacências a eliminação da redundância será quase de certeza uma má ideia. Se se representar a aresta (u,v) \(\in\) E apenas por um nó, na lista de adjacência de u ou de v, então, para ter acesso a todos os vértices adjacentes a um qualquer nó não bastará percorrer a sua lista de adjacências; será necessário percorrer todas as listas de adjacências do grafo.
\end{itemize}

\subsection{Codigo}
\label{sec:org7530d9d}
\begin{verbatim}

#include <stdio.h>
#include <stdlib.h>

#define N 8

enum search { DFirst, BFirst };

typedef struct aresta {
  int destino, peso;
  struct aresta *prox;
} * LAdj;

typedef LAdj Grafo[N];

/* Adicionar uma aresta */
Ladj newA(int dest, int peso, Ladj t){
 Ladj new = malloc(sizeof(struct aresta));
 new->destino = dest ; 
 new->peso = peso;
 new->prox = t; 
 return new;
}

/* Construir um Grafo a partir de uma matriz */
void constroiGrafo(int mat[N][N], Grafo g) {
  for (int i = 0; i < N; i++) {
    g[i] = NULL;
    for (int j = 0; j < N; j++)
      if (mat[i][j] != 0)
        g[i] = newA(j, mat[i][j], g[i]);
  }
}

/* Determinar quantas arestas tem um determinado grafo */
int quantasArestas(Grafo g) {
  int n = 0;
  LAdj a;
  for (int i = 0; i < N; i++)
    for (a = g[i]; a; a = a->prox, n++)
      ;
  return n;
}

/* Determinar a capacidade de um grafo */
int capacidade(Grafo g, int v) {
  int n = 0;
  LAdj a;
  for (int i = 0; i < N; i++)
    for (a = g[i]; a; a = a->prox) {
      if (i == v)
        n -= a->peso;
      if (a->destino == v)
        n += a->peso;
    }
  return n;
}
\end{verbatim}
\captionof{figure}{\label{fig:Functions}Some functions}
\section{Algoritmos de Travessia de Grafos}
\label{sec:org364f273}
\end{document}
